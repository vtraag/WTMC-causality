\documentclass[aspectratio=169,notes=hide,compress]{beamer}
\usetheme{CWTS}

\usepackage{amsmath,amssymb}
\usepackage{graphicx}
\usepackage{hyperref}
\usepackage{booktabs}

\usepackage{tikz}
\usetikzlibrary{arrows.meta}
\usetikzlibrary{arrows}
\tikzset{>=latex}
\usetikzlibrary{positioning}
\usetikzlibrary{shapes.misc}
\usetikzlibrary{shapes.multipart}

\usepackage{pgfplots}
\pgfplotsset{compat=1.17}

\usepgfplotslibrary{colorbrewer}

\usepackage{marvosym}

\DeclareMathOperator{\doOp}{do}

\pgfplotsset{colormap/Set1-8}

\graphicspath{%
  {../../../traveller/presentations/logos/}%
  {figures/}%
}

\title{Causal diagrams}
\author{Vincent Traag}
\date{11 February 2021}
\institute{WTMC method track -- Computational methods \& modelling}

\setbeamercovered{invisible}

%\includeonlyframes{topic}
\begin{document}

\frame{\titlepage}

\frame{\tableofcontents}

\section{Observing $\neq$ doing}

\begin{frame}[t]{Predicting}

  \begin{itemize}
    \item<+-> Predict future citations
    \begin{itemize}
      \item<+-> Tweets predict future citations
    \end{itemize}
    \item<+-> Predict accept/reject decisions
    \begin{itemize}
      \item<+-> Early publication record predicts acceptance
    \end{itemize}
    \item<+-> Predict future career paths
    \begin{itemize}
      \item<+-> Status of PhD supervisor predicts future career
    \end{itemize}
    \item<+-> Predict breakthrough papers
    \begin{itemize}
      \item<+-> Multidisciplinary citations predict breakthrough papers
    \end{itemize}
  \end{itemize}

  \vskip1cm
  \uncover<+->{
    \begin{center}
      Observations $\rightarrow$ prediction
    \end{center}
  }
\end{frame}

\begin{frame}{Observation}

  \centering
  \includegraphics[width=0.5\textwidth]{xkcd-correlation}

\end{frame}

\begin{frame}[c]{Observing vs doing}

  \begin{block}{Prediction}
    Observing $x$ may make $y$ more likely to be observed.
  \end{block}

  \vskip2cm
  \pause
  \begin{block}{Intervention}
    \emph{Doing} $x$ may make $y$ more likely to be observed.
  \end{block}

\end{frame}

\tikzstyle{n}=[rectangle,draw=white,fill=Set1-B!50]
\begin{frame}[c]{Observation vs doing}

  \begin{itemize}
    \item Predict future citations
    \begin{itemize}
      \item Tweets predict future citations
      \item<2-> Does tweeting your paper increase citations?
    \end{itemize}
  \end{itemize}
  \vskip2cm

  \begin{overlayarea}{\textwidth}{3cm}
    \begin{center}
      \begin{tikzpicture}
        \draw[opacity=0] (-4,-1) grid (3,3);
        \only<3->{
          \node[n] (x) at (180:2) {Tweets};
          \node[n] (y) at (0:2) {Citations};
        }

        \only<3>{
          \draw[->] (x) edge (y);
        }

        \only<4>{
          \node[n] (z) at (90:2) {``Value''};
          \draw[->] (z) edge (x)
                    edge (y);
        }
      \end{tikzpicture}
    \end{center}
  \end{overlayarea}
\end{frame}

\begin{frame}[c]{Observation vs doing}

  \begin{itemize}
    \item Predict accept/reject decisions
    \begin{itemize}
      \item Early publication record predicts acceptance
      \item<2-> Does increasing the publication record make an ``accept'' more likely?
    \end{itemize}
  \end{itemize}
  \vskip2cm

  \begin{overlayarea}{\textwidth}{3cm}
    \begin{center}
      \begin{tikzpicture}
        \draw[opacity=0] (-4,-1) grid (3,3);
        \only<3->{
          \node[n] (x) at (180:2) {Publication record};
          \node[n] (y) at (0:2) {``Accept''};
        }

        \only<3>{
          \draw[->] (x) edge (y);
        }

        \only<4>{
          \node[n] (z) at (90:2) {``Skill''};
          \draw[->] (z) edge (x)
                    edge (y);
        }
      \end{tikzpicture}
    \end{center}
  \end{overlayarea}
\end{frame}

\begin{frame}{Causal inference}

  \begin{block}{Causality}
    \begin{itemize}
      \item Causality is about \emph{doing} something, about (hypothetical) interventions.
      \item Important for policy implications:
        \begin{itemize}
          \item If you observe $x$ and $y$ are associated, but they are \emph{not} causally related, then \emph{doing} $x$ does not change $y$, and $x$ should not be advocated as a policy to achieve a change in $y$.
        \end{itemize}
    \end{itemize}
  \end{block}

  \pause

  How to discern association from causation?

  \pause

  \begin{block}{Causal inference}
    \begin{enumerate}
      \item<+-> Assume a causal model (background knowledge, earlier research, \ldots).
      \item<+-> Given causal model, under some conditions associations may reveal \emph{causal} effects.
      \begin{itemize}
        \item Key to this is understanding $d$-separation.
      \end{itemize}
      \item<+-> Test whether implications of causal model hold in observations.
      \begin{itemize}
        \item Different causal models may have same observational implications.
      \end{itemize}
    \end{enumerate}
  \end{block}

\end{frame}

\section{$d$-separation simplified}

\tikzstyle{inline}=[anchor=base,rectangle,rounded corners=5pt,
                    text height=1.25ex,text depth=.25ex,
                    outer sep=0pt,inner xsep=2pt]
\begin{frame}{$d$-separation}

  \centering
  Two variables $x$ and $y$ are associated if between them there is an \\
  \only<1>{
    \tikz[baseline] { \node[inline,fill=white] {open}; }\!
    \tikz[baseline] { \node[inline,fill=white] {undirected path}; }
  }
  \only<2>{
    \tikz[baseline] { \node[inline,fill=white] {open}; }\!
    \tikz[baseline] { \node[inline,fill=Set1-A!20] {undirected path}; }
  }
  \only<3->{
    \tikz[baseline] { \node[inline,fill=Set1-B!20] {open}; }\!
    \tikz[baseline] { \node[inline,fill=Set1-A!20] {undirected path}; }
  }

  \vskip1cm

  \uncover<4->{
  \begin{tabular}{lll}
    Technical term & Meaning & Implication \\
    \toprule
    $d$-connected & there is an open undirected path & $x$ and $y$ are associated  \\
    $d$-separated & there is no open undirected path & $x$ and $y$ are not associated
  \end{tabular}
  }
\end{frame}

\tikzstyle{faded}=[opacity=0.3]
\tikzstyle{n2}=[n,fill=Set1-A!20]
\begin{frame}{$d$-separation -- Undirected paths}

  \begin{center}
    \begin{tikzpicture}
      \draw[opacity=0] (-1,-1) grid (7,2);
      \node[n,faded] (r) at (0,-0.5)  {Rigour};
      \node[n,faded] (p) at (1.5,1)   {Published};
      \node[n,faded] (n) at (4.5,1)   {Novelty};
      \node[n,faded] (m) at (6,-0.5)  {News mentions};
      \node[n,faded] (a) at (2.5, -2) {Author reputation};
      \node[n,faded] (t) at (8,-2)    {TV appearance};

      \draw[opacity=0.3,->]
        (r) edge (p)
        (n) edge (p)
            edge (m)
        (m) edge (t)
        (a) edge (p)
            edge (m);

      \only<2>{
        \node[n2] (a) at (2.5, -2) {Author reputation};
        \node[n] (m) at (6,-0.5)  {News mentions};
        \node[n2] (t) at (8,-2)    {TV appearance};

        \draw[very thick,->]
          (m) edge (t)
          (a) edge (m);
      }

      \only<3>{
        \node[n2] (p) at (1.5,1)   {Published};
        \node[n] (n) at (4.5,1)   {Novelty};
        \node[n2] (m) at (6,-0.5)  {News mentions};

        \draw[very thick,->]
          (n) edge (m)
              edge (p);
      }

      \only<4>{
        \node[n2] (r) at (0,-0.5)  {Rigour};
        \node[n] (p) at (1.5,1)   {Published};
        \node[n] (n) at (4.5,1)   {Novelty};
        \node[n2] (m) at (6,-0.5)  {News mentions};

        \draw[very thick,->]
          (r) edge (p)
          (n) edge (p)
              edge (m);
      }
    \end{tikzpicture}
  \end{center}

\end{frame}

\begin{frame}[c]{$d$-separation -- Open paths \& nodes}

  \centering
  An undirected path is open if \emph{all} nodes on the path are\!
  \only<1>{\tikz[baseline] { \node[inline,outer xsep=0pt,fill=white] {open}; } }
  \only<2>{\tikz[baseline] { \node[inline,outer xsep=0pt,fill=Set1-A!20] {open}; } }\!\!\!.

\end{frame}

\tikzstyle{open}=[n,fill=white]
\tikzstyle{closed}=[n,fill=black,text=white]
\begin{frame}{$d$-separation -- Mediators, confounders \& colliders}

  \begin{overlayarea}{\textwidth}{2cm}
    \centering
    \only<2>{Mediator (causation) -- Open}
    \only<3>{Confounder -- Open}
    \only<4>{Collider -- Closed}
    \vskip5mm

    \begin{tikzpicture}[x=2cm]
      \only<2->{
        \node (x) at (-1, 0) {$\cdots$};
        \node[n,open] (z) at ( 0, 0) {$z$};
        \node (y) at ( 1, 0) {$\cdots$};
      }
      \only<2>{
        \draw[->] (x) edge (z)
                  (z) edge (y);
      }
      \only<3>{
        \draw[->] (z) edge (x)
                      edge (y);
      }
      \only<4>{
        \node[n,closed] (z) at ( 0, 0) {$z$};

        \draw[->] (x) edge (z)
                  (y) edge (z);
      }
    \end{tikzpicture}
  \end{overlayarea}

  \begin{center}
    \begin{tikzpicture}
      \draw[opacity=0] (-1,-1) grid (7,2);
      \node[n,faded] (r) at (0,-0.5)  {Rigour};
      \node[n,faded] (p) at (1.5,1)   {Published};
      \node[n,faded] (n) at (4.5,1)   {Novelty};
      \node[n,faded] (m) at (6,-0.5)  {News mentions};
      \node[n,faded] (a) at (2.5, -2) {Author reputation};
      \node[n,faded] (t) at (8,-2)    {TV appearance};

      \draw[opacity=0.3,->]
        (r) edge (p)
        (n) edge (p)
            edge (m)
        (m) edge (t)
        (a) edge (p)
            edge (m);

      \only<2>{
        \node[n] (a) at (2.5, -2) {Author reputation};
        \node[n,open] (m) at (6,-0.5)  {News mentions};
        \node[n] (t) at (8,-2)    {TV appearance};

        \draw[very thick,->]
          (m) edge (t)
          (a) edge (m);
      }

      \only<3>{
        \node[n] (p) at (1.5,1)   {Published};
        \node[n] (m) at (6,-0.5)  {News mentions};
        \node[n,open] (n) at (4.5,1)   {Novelty};

        \draw[very thick,->]
          (n) edge (m)
              edge (p);
      }

      \only<4>{
        \node[n] (r) at (0,-0.5)  {Rigour};
        \node[n] (m) at (6,-0.5)  {News mentions};
        \node[n,closed] (p) at (1.5,1)   {Published};
        \node[n,open] (n) at (4.5,1)   {Novelty};

        \draw[very thick,->]
          (r) edge (p)
          (n) edge (p)
              edge (m);
      }
    \end{tikzpicture}
  \end{center}

\end{frame}

\begin{frame}[c]{Conditioning on variables}

  Many possible ways to ``condition'' on a variable:
  \begin{itemize}
    \item Include only subset of cases in analysis.
    \item Perform analyses separately on different subsets of data.
    \item ``Normalise'' indicators (e.g field-normalise).
    \item Include variable in regression.
  \end{itemize}

\end{frame}

\tikzstyle{conditioned}=[rounded rectangle,draw=white,double=black]
\begin{frame}[c]{$d$-separation -- Open nodes \& conditioning}

  \centering
  Conditioning on a variable ``flips'' the node on a path \\
  from open
  \tikz[baseline] { \node[anchor=south,n,open,draw=black] {}; }
  to closed
  \tikz[baseline] { \node[anchor=south,n,closed] {}; }
  and vice-versa.

  \vskip1.5cm

  \begin{tabular}{lcc}
     & Unconditioned & Conditioned \\[3mm]
    Mediator &
    \tikz{
      \node (x) at (-1, 0) {$\cdots$};
      \node[n,open] (z) at ( 0, 0) {$z$};
      \node (y) at ( 1, 0) {$\cdots$};

      \draw[->] (x) edge (z)
                (z) edge (y);}
      &
      \tikz{
        \node (x) at (-1, 0) {$\cdots$};
        \node[n,closed,conditioned] (z) at ( 0, 0) {$z$};
        \node (y) at ( 1, 0) {$\cdots$};

        \draw[->] (x) edge (z)
                  (z) edge (y);} \\[5mm]

      Confounder &
      \tikz{
        \node (x) at (-1, 0) {$\cdots$};
        \node[n,open] (z) at ( 0, 0) {$z$};
        \node (y) at ( 1, 0) {$\cdots$};

        \draw[->] (z) edge (x)
                  (z) edge (y);}
        &
        \tikz{
          \node (x) at (-1, 0) {$\cdots$};
          \node[n,closed,conditioned] (z) at ( 0, 0) {$z$};
          \node (y) at ( 1, 0) {$\cdots$};

          \draw[->] (z) edge (x)
                    (z) edge (y);} \\[5mm]

        Collider &
        \tikz{
          \node (x) at (-1, 0) {$\cdots$};
          \node[n,closed] (z) at ( 0, 0) {$z$};
          \node (y) at ( 1, 0) {$\cdots$};

          \draw[->] (x) edge (z)
                    (y) edge (z);}
          &
          \tikz{
            \node (x) at (-1, 0) {$\cdots$};
            \node[n,open,conditioned] (z) at ( 0, 0) {$z$};
            \node (y) at ( 1, 0) {$\cdots$};

            \draw[->] (x) edge (z)
                      (y) edge (z);} \\[5mm]
  \end{tabular}

\end{frame}

\begin{frame}{$d$-separation -- Conditioning}
  \begin{overlayarea}{\textwidth}{2cm}
    \centering
    \only<2>{Condition on mediator -- Closed}
    \only<3>{Condition on confounder -- Closed}
    \only<4>{Condition on collider -- Open}
    \only<6->{Conditioning may close one path but open another.}
    \vskip5mm

    \begin{tikzpicture}[x=2cm]
      \only<2-4>{
        \node (x) at (-1, 0) {$\cdots$};
        \node[n,closed,conditioned] (z) at ( 0, 0) {$z$};
        \node (y) at ( 1, 0) {$\cdots$};
      }
      \only<2>{
        \draw[->] (x) edge (z)
                  (z) edge (y);
      }
      \only<3>{
        \draw[->] (z) edge (x)
                      edge (y);
      }
      \only<4>{
        \node[n,open,conditioned] (z) at ( 0, 0) {$z$};

        \draw[->] (x) edge (z)
                  (y) edge (z);
      }
    \end{tikzpicture}
  \end{overlayarea}

  \begin{center}
    \begin{tikzpicture}
      \draw[opacity=0] (-1,-1) grid (7,2);
      \node[n,faded] (r) at (0,-0.5)  {Rigour};
      \node[n,faded] (p) at (1.5,1)   {Published};
      \node[n,faded] (n) at (4.5,1)   {Novelty};
      \node[n,faded] (m) at (6,-0.5)  {News mentions};
      \node[n,faded] (a) at (2.5, -2) {Author reputation};
      \node[n,faded] (t) at (8,-2)    {TV appearance};

      \draw[opacity=0.3,->]
        (r) edge (p)
        (n) edge (p)
            edge (m)
        (m) edge (t)
        (a) edge (p)
            edge (m);

      \only<2>{
        \node[n] (a) at (2.5, -2) {Author reputation};
        \node[n,closed,conditioned] (m) at (6,-0.5)  {News mentions};
        \node[n] (t) at (8,-2)    {TV appearance};

        \draw[very thick,->]
          (m) edge (t)
          (a) edge (m);
      }

      \only<3>{
        \node[n] (p) at (1.5,1)   {Published};
        \node[n] (m) at (6,-0.5)  {News mentions};
        \node[n,closed,conditioned] (n) at (4.5,1)   {Novelty};

        \draw[very thick,->]
          (n) edge (m)
              edge (p);
      }

      \only<4>{
        \node[n] (r) at (0,-0.5)  {Rigour};
        \node[n] (m) at (6,-0.5)  {News mentions};
        \node[n,open,conditioned] (p) at (1.5,1)   {Published};
        \node[n,open] (n) at (4.5,1)   {Novelty};

        \draw[very thick,->]
          (r) edge (p)
          (n) edge (p)
              edge (m);
      }

      \only<5>{
        \node[n] (r) at (0,-0.5)  {Rigour};
        \node[n] (m) at (6,-0.5)  {News mentions};
        \node[n,open,conditioned] (p) at (1.5,1)   {Published};
        \node[n,closed,conditioned] (n) at (4.5,1)   {Novelty};

        \draw[very thick,->]
          (r) edge (p)
          (n) edge (p)
              edge (m);
      }

      \only<6>{
        \node[n] (a) at (2.5, -2) {Author reputation};
        \node[n,closed,conditioned] (m) at (6,-0.5)  {News mentions};
        \node[n] (t) at (8,-2)    {TV appearance};

        \draw[very thick,->]
          (m) edge (t)
          (a) edge (m);
      }

      \only<7>{
        \node[n] (a) at (2.5, -2) {Author reputation};
        \node[n,open,conditioned] (m) at (6,-0.5)  {News mentions};
        \node[n] (n) at (4.5,1)   {Novelty};

        \draw[very thick,->]
          (a) edge (m)
          (n) edge (m);
      }
    \end{tikzpicture}
  \end{center}

\end{frame}

  % \begin{tikzpicture}
  %   \node[n,closed,conditioned] at (0,0) {$z$};
  %   \begin{scope}
  %     \clip (-1,0) rectangle (1,1);
  %     \node[n,open,conditioned] at (0,0) {$z$};
  %   \end{scope}
  % \end{tikzpicture}

\begin{frame}[c]{Conditioning on descendants and $d$-separation}

  \centering
  Conditioning on descendant of collider opens it.

  \vskip1.5cm

  \begin{tabular}{lccc}
     & Unconditioned & Conditioned & Descendant conditioning \\[3mm]
    Mediator &
    \tikz{
      \node (x) at (-1, 0) {$\cdots$};
      \node[n,open] (z) at ( 0, 0) {$z$};
      \node (y) at ( 1, 0) {$\cdots$};

      \draw[->] (x) edge (z)
                (z) edge (y);}
      &
      \tikz{
        \node (x) at (-1, 0) {$\cdots$};
        \node[n,closed,conditioned] (z) at ( 0, 0) {$z$};
        \node (y) at ( 1, 0) {$\cdots$};

        \draw[->] (x) edge (z)
                  (z) edge (y);} \\[5mm]

      Confounder &
      \tikz{
        \node (x) at (-1, 0) {$\cdots$};
        \node[n,open] (z) at ( 0, 0) {$z$};
        \node (y) at ( 1, 0) {$\cdots$};

        \draw[->] (z) edge (x)
                  (z) edge (y);}
        &
        \tikz{
          \node (x) at (-1, 0) {$\cdots$};
          \node[n,closed,conditioned] (z) at ( 0, 0) {$z$};
          \node (y) at ( 1, 0) {$\cdots$};

          \draw[->] (z) edge (x)
                    (z) edge (y);} \\[5mm]

        Collider &
        \tikz{
          \node (x) at (-1, 0) {$\cdots$};
          \node[n,closed] (z) at ( 0, 0) {$z$};
          \node (y) at ( 1, 0) {$\cdots$};

          \draw[->] (x) edge (z)
                    (y) edge (z);}
          &
          \tikz{
            \node (x) at (-1, 0) {$\cdots$};
            \node[n,open,conditioned] (z) at ( 0, 0) {$z$};
            \node (y) at ( 1, 0) {$\cdots$};

            \draw[->] (x) edge (z)
                      (y) edge (z);}

          &
          \tikz{
            \node (x) at (-1, 0) {$\cdots$};

            \node[n,open] at (0,0) {$z$};
            \node[n,conditioned] (w) at (0,0.8) {$\cdots$};

            \node (y) at ( 1, 0) {$\cdots$};

            \draw[->] (x) edge (z)
                      (y) edge (z)
                      (z) edge (w);}
  \end{tabular}

\end{frame}

\begin{frame}{$d$-separation -- Conditioning on descendants}
  \begin{overlayarea}{\textwidth}{2cm}
    \centering
    \only<2>{
      Condition on descendant of collider -- Opens collider

      \vskip5mm

      \begin{tikzpicture}[x=2cm]
          \node (x) at (-1, 0) {$\cdots$};

          \node[n,open] at (0,0) {$z$};
          \node[n,conditioned] (w) at (0,0.8) {$\cdots$};

          \node (y) at ( 1, 0) {$\cdots$};

          \draw[->] (x) edge (z)
                    (y) edge (z)
                    (z) edge (w);
      \end{tikzpicture}
    }
  \end{overlayarea}

  \begin{center}
    \begin{tikzpicture}
      \draw[opacity=0] (-1,-1) grid (7,2);
      \node[n,faded] (r) at (0,-0.5)  {Rigour};
      \node[n,faded] (p) at (1.5,1)   {Published};
      \node[n,faded] (n) at (4.5,1)   {Novelty};
      \node[n,faded] (m) at (6,-0.5)  {News mentions};
      \node[n,faded] (a) at (2.5, -2) {Author reputation};
      \node[n,faded] (t) at (8,-2)    {TV appearance};

      \draw[opacity=0.3,->]
        (r) edge (p)
        (n) edge (p)
            edge (m)
        (m) edge (t)
        (a) edge (p)
            edge (m);

      \only<2>{
        \node[n] (a) at (2.5, -2) {Author reputation};
        \node[n,open] (m) at (6,-0.5)  {News mentions};
        \node[n] (n) at (4.5,1)   {Novelty};
        \node[n,conditioned] (t) at (8,-2)    {TV appearance};

        \draw[very thick,->]
          (a) edge (m)
          (n) edge (m);
      }
    \end{tikzpicture}
  \end{center}

\end{frame}

\begin{frame}{$d$-separation summary}

  \begin{enumerate}
    \item<+-> $x$ and $y$ are associated if there is an open undirected path between them.
    \item<+-> An undirected path is open if all nodes are open.
    \item<+-> A node on a path is open
    \tikz[baseline] { \node[anchor=south,n,open,draw=black] {}; }
    or closed
    \tikz[baseline] { \node[anchor=south,n,closed] {}; } in the following situations
    \begin{itemize}
      \item Conditioning on a variable ``flips'' nodes from open to closed and vice-versa.
    \end{itemize}
  \end{enumerate}

  \vskip1cm

  \uncover<3->{
  \begin{tabular}{lccc}
    & Unconditioned & Conditioned & Descendant conditioning \\[3mm]
   Mediator &
   \tikz{
     \node (x) at (-1, 0) {$\cdots$};
     \node[n,open] (z) at ( 0, 0) {$z$};
     \node (y) at ( 1, 0) {$\cdots$};

     \draw[->] (x) edge (z)
               (z) edge (y);}
     &
     \tikz{
       \node (x) at (-1, 0) {$\cdots$};
       \node[n,closed,conditioned] (z) at ( 0, 0) {$z$};
       \node (y) at ( 1, 0) {$\cdots$};

       \draw[->] (x) edge (z)
                 (z) edge (y);} \\[5mm]

     Confounder &
     \tikz{
       \node (x) at (-1, 0) {$\cdots$};
       \node[n,open] (z) at ( 0, 0) {$z$};
       \node (y) at ( 1, 0) {$\cdots$};

       \draw[->] (z) edge (x)
                 (z) edge (y);}
       &
       \tikz{
         \node (x) at (-1, 0) {$\cdots$};
         \node[n,closed,conditioned] (z) at ( 0, 0) {$z$};
         \node (y) at ( 1, 0) {$\cdots$};

         \draw[->] (z) edge (x)
                   (z) edge (y);} \\[5mm]

       Collider &
       \tikz{
         \node (x) at (-1, 0) {$\cdots$};
         \node[n,closed] (z) at ( 0, 0) {$z$};
         \node (y) at ( 1, 0) {$\cdots$};

         \draw[->] (x) edge (z)
                   (y) edge (z);}
         &
         \tikz{
           \node (x) at (-1, 0) {$\cdots$};
           \node[n,open,conditioned] (z) at ( 0, 0) {$z$};
           \node (y) at ( 1, 0) {$\cdots$};

           \draw[->] (x) edge (z)
                     (y) edge (z);}

         &
         \tikz{
          \node (x) at (-1, 0) {$\cdots$};

          \node[n,open] at (0,0) {$z$};
          \node[n,conditioned] (w) at (0,0.8) {$\cdots$};

          \node (y) at ( 1, 0) {$\cdots$};

          \draw[->] (x) edge (z)
                    (y) edge (z)
                    (z) edge (w);}
 \end{tabular}
  }
\end{frame}

\begin{frame}{Further reading}

  \centering
  \begin{columns}
    \begin{column}{0.5\textwidth}
      \centering
      \href{http://bayes.cs.ucla.edu/WHY/}{\includegraphics[height=7cm]{book-of-why-cover.jpeg}}
    \end{column}
    \begin{column}{0.5\textwidth}
      \centering
      \href{http://bayes.cs.ucla.edu/PRIMER/}{\includegraphics[height=7cm]{primer-cover.jpg}}
    \end{column}
  \end{columns}

\end{frame}

\end{document}

